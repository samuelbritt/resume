% options: font family ('sans' and 'roman'), standard article options
\documentclass[10pt,letterpaper,roman]{moderncv}

% moderncv themes (load first)
% options: casual, classic, oldstyle, banking
\moderncvstyle{classic}

 % options: blue, orange, green, red, purple, grey, black
% \moderncvcolor{orange}
\definecolor{color0}{rgb}{0,0,0}% black
% \definecolor{color1}{rgb}{0.95,0.55,0.15}% orange
% \definecolor{orange}{rgb}{0.95,0.55,0.15}% orange
% \colorlet{color1}{orange!90!black}
\definecolor{color1}{rgb}{1,.5,0}
\definecolor{color2}{gray}{0.4}% dark grey

\usepackage[osf]{mathpazo}
\usepackage{microtype}
\usepackage{etoolbox}
\usepackage{relsize}
\usepackage{siunitx}
\usepackage[super]{nth}
\usepackage[hscale=0.85,top=.4in,bottom=.3in]{geometry}

% Toggles for various versions. See etoolbox documentation.
\newtoggle{engineering}

% The version
\toggletrue{engineering}

%  default font; use '\sfdefault' for sans serif, '\rmdefault' for
%  roman, or any tex font name
% \renewcommand{\familydefault}{\sfdefault}

% Small caps headings
\renewcommand*\sectionfont{%
  \scshape \lsstyle \fontsize{14}{24}\selectfont%
}

% Italics, not slanted
\renewcommand*{\titlefont}{\large\mdseries\itshape}
\renewcommand*{\addressfont}{\small\mdseries\itshape}

% Pretty ampersand
\let\oldamp\&
\renewcommand\&{{\itshape \oldamp}}

% Pretty Languages
\newcommand\Matlab{\textsc{Matlab}}
\newcommand\Fortran{\textsc{Fortran}}
\newcommand\Unix{\textsc{Unix}}
\newcommand\CC{C%
  \nolinebreak[4]\hspace{-.05em}\raisebox{.29ex}{\relsize{-1}{++}}%
}


% uncomment to suppress page numbering for CVs longer than one page
\nopagenumbers{}

% width of date column
\setlength{\hintscolumnwidth}{1.9cm}

% lists use dashes instead of open bullets
% \let\labelitemi\labelitemii

% Short description
\newcommand\cvdesc[2]{%
  \cventry{}%
  {#1}%
  {}%
  {\small #2}%
  {}%
  {}%
}

% Education command
% 1: date
% 2: Degree (M.S., B.S., etc)
% 3: Major
% 4: GPA
% 5: School
% 6: Description
\newcommand\cvedu[6]
  {
    \cventry{#1}
      {#2}
      {#3\hfill\small\upshape{#4}}
      {}
      {}
      {#5
      \ifthenelse{\equal{#6}{}}{}{\\#6}
      }
  }


% Copied from moderncvclassic.cls for the sole purpose of using \itshape
% instead of \slshape in the location slot.
\renewcommand*{\cventry}[7][.25em]{%
  \cvitem[#1]{#2}{%
    {\bfseries#3}%
    \ifthenelse{\equal{#4}{}}{}{, {\itshape#4}}%
    \ifthenelse{\equal{#5}{}}{}{, #5}%
    \ifthenelse{\equal{#6}{}}{}{, #6}%
    .\strut%
    \ifx&#7&%
    \else{\newline{}\begin{minipage}[t]{\linewidth}\small#7\end{minipage}}\fi}%
}

% Github and Linkedin footer items
% from http://tex.stackexchange.com/questions/70216/adding-sections-such-as-linkedin-and-github-to-a-moderncv-footer
\usepackage{tikz}

\newcommand{\github}{%
\begin{tikzpicture}[y=0.15pt, x=0.15pt,yscale=-1, inner sep=0pt, outer sep=0pt,opacity=0.4]
  \begin{scope}[shift={(506.69823,386.92617)}]
    \path[fill=gray] (116.9933,59.7217) .. controls (116.9933,71.2283) and
      (107.6655,80.5562) .. (96.1589,80.5562) .. controls (84.6524,80.5562) and
      (75.3245,71.2283) .. (75.3245,59.7217) .. controls (75.3245,48.2152) and
      (84.6524,38.8873) .. (96.1589,38.8873) .. controls (107.6654,38.8873) and
      (116.9933,48.2152) .. (116.9933,59.7217) -- cycle;
    \path[cm={{0.88462,0.0,0.0,0.88462,(11.09526,6.89097)}},fill=white]
      (116.9933,59.7217) .. controls (116.9933,71.2283) and (107.6655,80.5562) ..
      (96.1589,80.5562) .. controls (84.6524,80.5562) and (75.3245,71.2283) ..
      (75.3245,59.7217) .. controls (75.3245,48.2152) and (84.6524,38.8873) ..
      (96.1589,38.8873) .. controls (107.6654,38.8873) and (116.9933,48.2152) ..
      (116.9933,59.7217) -- cycle;
    \path[fill=gray,nonzero rule] (103.4671,45.2878) .. controls (102.9322,45.4374)
      and (101.2003,46.2576) .. (100.5403,46.6739) -- (100.1099,46.9454) --
      (99.4882,46.8019) .. controls (99.0810,46.7080) and (98.1204,46.6415) ..
      (96.7048,46.6094) .. controls (94.4953,46.5593) and (93.4339,46.6361) ..
      (92.2380,46.9324) -- (91.6450,47.0793) -- (90.9468,46.6426) .. controls
      (90.0955,46.1101) and (88.7784,45.4948) .. (88.1825,45.3512) .. controls
      (87.9348,45.2916) and (87.5225,45.2429) .. (87.2643,45.2429) .. controls
      (86.8530,45.2429) and (86.7816,45.2733) .. (86.6817,45.4916) .. controls
      (86.3049,46.3144) and (86.1702,48.1697) .. (86.3982,49.3940) --
      (86.5087,49.9870) -- (86.0485,50.6088) .. controls (85.4184,51.4600) and
      (84.9876,52.3958) .. (84.8509,53.2104) .. controls (84.6439,54.4443) and
      (84.8398,57.3849) .. (85.1880,58.2702) .. controls (85.2564,58.4443) and
      (85.2939,58.4403) .. (81.6976,58.6338) .. controls (79.2203,58.7672) and
      (77.4880,58.9815) .. (77.2948,59.1788) .. controls (77.1683,59.3080) and
      (77.2021,59.3161) .. (77.6325,59.2604) .. controls (79.8802,58.9695) and
      (83.0680,58.7293) .. (84.6818,58.7293) .. controls (85.3322,58.7293) and
      (85.3437,58.7337) .. (85.4709,59.0402) .. controls (85.5424,59.2123) and
      (85.5936,59.3574) .. (85.5857,59.3654) .. controls (85.5778,59.3733) and
      (84.8826,59.4288) .. (84.0409,59.4888) .. controls (82.1375,59.6245) and
      (80.3024,59.8884) .. (78.6942,60.2577) .. controls (77.5177,60.5279) and
      (77.1884,60.6573) .. (77.3264,60.7953) .. controls (77.3578,60.8267) and
      (77.9386,60.7190) .. (78.6081,60.5575) .. controls (80.6932,60.0548) and
      (83.4463,59.6858) .. (85.1122,59.6858) .. controls (85.7817,59.6858) and
      (85.8050,59.6938) .. (85.9497,59.9727) .. controls (86.1509,60.3606) and
      (87.1973,61.4638) .. (87.6756,61.7923) .. controls (88.7575,62.5354) and
      (90.1146,63.0487) .. (91.7311,63.3262) .. controls (92.3241,63.4280) and
      (92.8529,63.5117) .. (92.9028,63.5117) .. controls (92.9519,63.5117) and
      (92.8171,63.7221) .. (92.6084,63.9708) .. controls (92.2151,64.4395) and
      (91.8427,65.1574) .. (91.8393,65.4534) .. controls (91.8343,65.8877) and
      (90.1911,66.2247) .. (89.1390,66.0071) .. controls (88.4365,65.8618) and
      (87.9449,65.5203) .. (87.3370,64.7552) .. controls (86.5997,63.8274) and
      (86.0013,63.2318) .. (85.6000,63.0268) .. controls (85.1313,62.7874) and
      (84.1718,62.7744) .. (83.9782,63.0048) .. controls (83.8657,63.1387) and
      (83.8975,63.1954) .. (84.2322,63.4586) .. controls (85.1908,64.2122) and
      (85.6680,64.7934) .. (86.1681,65.8169) .. controls (86.7336,66.9742) and
      (87.2885,67.5731) .. (88.1825,67.9913) .. controls (88.6992,68.2330) and
      (88.8042,68.2463) .. (90.1911,68.2463) -- (91.6546,68.2463) --
      (91.6259,70.0923) -- (91.5972,71.9383) -- (91.2050,72.2922) .. controls
      (90.9850,72.4908) and (90.6785,72.7603) .. (90.5068,72.9061) .. controls
      (90.0483,73.2955) and (90.1529,73.4104) .. (90.9946,73.4418) .. controls
      (91.6450,73.4662) and (91.7691,73.4390) .. (92.3241,73.1503) .. controls
      (93.3630,72.6098) and (93.3667,72.5983) .. (93.3667,69.8628) .. controls
      (93.3667,67.3377) and (93.4455,66.7059) .. (93.8107,66.3047) --
      (94.0458,66.0464) -- (93.9980,69.2506) .. controls (93.9695,71.1540) and
      (93.9075,72.6024) .. (93.8449,72.8183) .. controls (93.7868,73.0192) and
      (93.6134,73.3252) .. (93.4575,73.5022) .. controls (93.3059,73.6744) and
      (93.1754,73.9155) .. (93.1754,74.0235) .. controls (93.1754,74.1976) and
      (93.2328,74.2243) .. (93.6058,74.2243) .. controls (94.3519,74.2243) and
      (95.3191,73.5586) .. (95.6209,72.8374) .. controls (95.8285,72.3417) and
      (95.9492,70.6280) .. (95.9492,68.1794) -- (95.9492,65.9029) --
      (96.4179,65.9029) -- (96.4465,69.1311) .. controls (96.4752,72.3544) and
      (96.4756,72.3599) .. (96.7144,72.8374) .. controls (97.1209,73.6505) and
      (98.5189,74.4873) .. (99.0195,74.2173) .. controls (99.2785,74.0776) and
      (99.2470,73.9374) .. (98.8154,73.3061) .. controls (98.5996,72.9905) and
      (98.3935,72.5452) .. (98.3372,72.2731) .. controls (98.2088,71.6514) and
      (98.2544,66.1949) .. (98.3882,66.1752) .. controls (98.4417,66.1673) and
      (98.5682,66.3047) .. (98.6752,66.4864) .. controls (98.8508,66.7849) and
      (98.8704,67.0316) .. (98.9143,69.4898) .. controls (98.9477,71.3645) and
      (98.9985,72.2310) .. (99.0833,72.3783) .. controls (99.2883,72.7344) and
      (99.9568,73.2398) .. (100.3777,73.3570) .. controls (100.6002,73.4189) and
      (101.0568,73.4562) .. (101.4011,73.4406) .. controls (102.2046,73.4043) and
      (102.2524,73.2299) .. (101.5924,72.7428) .. controls (100.6531,72.0496) and
      (100.6840,72.1775) .. (100.6746,68.9637) .. controls (100.6656,65.9699) and
      (100.6109,65.4703) .. (100.2007,64.6499) .. controls (100.0812,64.4108) and
      (99.8134,64.0644) .. (99.5982,63.8704) -- (99.2108,63.5213) --
      (99.6603,63.4617) .. controls (100.5690,63.3414) and (102.0372,63.0328) ..
      (102.6446,62.8345) .. controls (104.1654,62.3382) and (105.5084,61.3208) ..
      (106.1445,60.1832) -- (106.4227,59.6858) -- (106.9679,59.6858) .. controls
      (108.9956,59.6858) and (112.7503,60.2177) .. (114.7632,60.7901) .. controls
      (114.9162,60.8337) and (114.9832,60.8090) .. (114.9832,60.7092) .. controls
      (114.9832,60.3420) and (111.4059,59.7105) .. (108.1061,59.4950) .. controls
      (107.2931,59.4419) and (106.6181,59.3838) .. (106.5996,59.3654) .. controls
      (106.5815,59.3473) and (106.6145,59.1932) .. (106.6713,59.0306) --
      (106.7765,58.7293) -- (107.9817,58.7323) .. controls (109.6496,58.7363) and
      (111.7789,58.8872) .. (113.5293,59.1252) .. controls (114.8684,59.3073) and
      (115.2129,59.3130) .. (115.0501,59.1502) .. controls (114.8456,58.9456) and
      (112.1137,58.6482) .. (109.3399,58.5285) .. controls (108.0008,58.4707) and
      (106.8944,58.4168) .. (106.8865,58.4089) .. controls (106.8785,58.4010) and
      (106.9394,58.0694) .. (107.0204,57.6772) .. controls (107.1184,57.2030) and
      (107.1719,56.3764) .. (107.1782,55.2382) .. controls (107.1862,53.7174) and
      (107.1624,53.4295) .. (106.9708,52.7704) .. controls (106.6953,51.8235) and
      (106.3173,51.0734) .. (105.7225,50.2931) -- (105.2557,49.6810) --
      (105.2940,48.0598) .. controls (105.3295,46.5581) and (105.3160,46.3927) ..
      (105.1123,45.8168) -- (104.8923,45.1951) -- (104.4140,45.1760) .. controls
      (104.1462,45.1653) and (103.7296,45.2145) .. (103.4671,45.2879) -- cycle;
  \end{scope}
\end{tikzpicture}
}

\newcommand{\linkedin}{%
\begin{tikzpicture}[y=0.15pt, x=0.15pt,yscale=-1, inner sep=0pt, outer sep=0pt,opacity=0.4]
  \begin{scope}[cm={{0.59444,0.0,0.0,0.59444,(346.38938,123.06674)}}]
    \path[fill=gray] (380.7408,201.6221) -- (434.0804,201.6221) .. controls
      (438.6572,201.6221) and (442.3417,205.3067) .. (442.3417,209.8835) --
      (442.3417,263.5823) .. controls (442.3417,268.1591) and (438.6572,271.8436) ..
      (434.0804,271.8436) -- (380.7408,271.8436) .. controls (376.1640,271.8436) and
      (372.4794,268.1591) .. (372.4794,263.5823) -- (372.4794,209.8835) .. controls
      (372.4794,205.3067) and (376.1640,201.6221) .. (380.7408,201.6221) -- cycle;
    \begin{scope}[xscale=0.981,yscale=1.019,fill=white]
      \path[fill=white] (402.5597,253.0812) -- (402.5597,223.9631) --
        (393.5086,223.9631) -- (393.5086,253.0812) -- cycle(398.0937,211.3394) ..
        controls (396.6162,211.3680) and (395.4476,211.8021) .. (394.5879,212.6419) ..
        controls (393.7282,213.4818) and (393.2891,214.5561) .. (393.2705,215.8649) ..
        controls (393.2879,217.1476) and (393.7146,218.2145) .. (394.5507,219.0655) ..
        controls (395.3868,219.9165) and (396.5281,220.3581) .. (397.9746,220.3904) ..
        controls (399.5067,220.3582) and (400.7001,219.9165) .. (401.5548,219.0655) ..
        controls (402.4095,218.2145) and (402.8437,217.1476) .. (402.8574,215.8649) ..
        controls (402.8152,214.5561) and (402.3785,213.4818) .. (401.5474,212.6419) ..
        controls (400.7162,211.8021) and (399.5650,211.3679) .. (398.0937,211.3394) --
        cycle;
      \path[fill=white] (409.7910,253.0812) -- (418.8420,253.0812) --
        (418.8420,236.2892) .. controls (418.8400,235.8674) and (418.8594,235.4605) ..
        (418.9015,235.0685) .. controls (418.9437,234.6765) and (419.0231,234.3291) ..
        (419.1397,234.0264) .. controls (419.4635,233.1556) and (420.0068,232.3815) ..
        (420.7698,231.7041) .. controls (421.5327,231.0268) and (422.5375,230.6695) ..
        (423.7843,230.6323) .. controls (425.4081,230.6609) and (426.5817,231.2439) ..
        (427.3049,232.3815) .. controls (428.0282,233.5190) and (428.3830,235.0400) ..
        (428.3693,236.9442) -- (428.3693,253.0812) -- (437.4203,253.0812) --
        (437.4203,235.8724) .. controls (437.3582,231.5975) and (436.3658,228.4316) ..
        (434.4430,226.3748) .. controls (432.5202,224.3180) and (430.0391,223.2958) ..
        (426.9998,223.3081) .. controls (424.5633,223.3851) and (422.6033,223.9309) ..
        (421.1196,224.9456) .. controls (419.6359,225.9604) and (418.5988,226.9826) ..
        (418.0083,228.0123) -- (417.8297,228.0123) -- (417.4129,223.9631) --
        (409.5528,223.9631) .. controls (409.6148,225.2695) and (409.6694,226.6911) ..
        (409.7165,228.2281) .. controls (409.7636,229.7652) and (409.7884,231.4399) ..
        (409.7909,233.2523) -- cycle;
    \end{scope}
  \end{scope}
\end{tikzpicture}
}

\newcommand\smallsymbol[1]{\raisebox{-1pt}{#1\hspace{0pt}}}
\newcommand\githubsymbol{\smallsymbol{\githubpic}}
\newcommand\github[1]{\githubsymbol\httplink{#1}}

\newcommand\linkedinsymbol{\smallsymbol{\linkedinpic}}
\newcommand\linkedin[1]{\linkedinsymbol\httplink{#1}}

% personal data
\firstname{Samuel}
\familyname{Britt}

% All the following are optional
\address{1467 Hembree Station Dr}{Marietta, GA 30062}
\email{samuelbritt@gmail.com}
\phone{(205)~515~0618}
\extrainfo{%
%   \github{github.com/samuelbritt}
%   \\ \footersymbol
  \linkedin{linkedin.com/in/samuelbritt}
}
% \mobile{(205)~515~0618}
% \fax{+3~(456)~789~012}
% \homepage{www.johndoe.com}
% % \photo[pic-height][frame-thickness]{pic-file}
% \photo[64pt][0.4pt]{picture}

% Optional
% % \quote{Some quote (optional)}
% \iftoggle{engineering}{
%   % \title{Materials Engineer -- Systems Developer}
%   \title{
%     \normalsize
%     From materials engineering to software development through a
%     passion for technology.
%   }
%   % \title{%\normalsize
%   %   Seeking to leverage an engineering background and passion for
%   %   technology.
%     % to pursue systems software development\\through a passion for
%     % technology.
%   % }
% }{}

\begin{document}
\makecvtitle
\vspace*{-3em}

\section{Education}
\cvedu{2011--2013}
  {Master of Science}
  {Computer Science}
  {GPA: 3.7}
  {Georgia Institute of Technology, Atlanta GA}
  {
    Specialization in Systems Software.
  }

\cvedu{2009--2011}
  {Graduate Research Assistant}
  {Materials Science \& Engineering, Mechanics of Materials}
  {GPA: 3.9}
  {Georgia Institute of Technology, Atlanta GA}
  {
    Modeling and simulation research in the mechanics of
    $\alpha$+$\beta$ titanium alloys.
  }

\cvedu{2004--2009}
  {Bachelor of Science, Highest Honors}
  {Materials Science \& Engineering}
  {GPA: 4.0}
  {Georgia Institute of Technology, Atlanta GA}
  {}

\section{Skills \& Technologies}
\cvitem{Proficient:}{C, Python (and SciPy), Perl, SQL, MS SQL Server, Git, Perforce, \Unix.}
\cvitem{Familiar:}{\CC, Java, JavaScript, \Matlab, Bash, Eclipse,
  Android development.}

\section{Experience}

\cventry{2015--present}
  {Staff Developer, People Manager}
  {athenahealth}
  {Atlanta, GA}
  {}
  {
    \begin{itemize}
      \item Led a team of 7 stateside and offshore developers in developing and
        operating a distributed ETL pipeline from an Oracle/Linux stack into
        centralized Microsoft SQL Server databases.
      \item Stabilized ETL operations, bringing monthly failure rates from
        around 9\% to less than 0.5\%.
      \item Developed a variety of internal tools to improve team productivity,
        leveraging shell scripts, Perl, and Backbone.js.
      \item Developed a data access API and SQL query-generating platform
        servicing BI applications.
      \item Selected to join a small Agile Scrum team developing the
        next-generation data analytics architecture for athena.
      \item Architected ETL into Snowflake, a cloud-based MPP database, and
        developed data models in the Looker modeling and visualization tool.
    \end{itemize}
  }


\cventry{2014--2015}
  {Senior Developer}
  {athenahealth}
  {Atlanta, GA}
  {}
  {
    \begin{itemize}
      \item Developed a T-SQL-based data transformation framework to allow for
        rapid extensions of existing data pipelines.
      \item Leveraged the framework to rapidly deploy a transformation of our
        largest data pipelines into a columnar database format, dramatically
        reducing runtimes for analytics queries.
    \end{itemize}
  }

\cventry{2013--2014}
  {Developer}
  {athenahealth}
  {Atlanta, GA}
  {}
  {
    \begin{itemize}
      \item Data customization and ETL process automation for Analytics, a
        health care revenue cycle business intelligence tool targeted at large
          health systems. Utilized the Microsoft BI stack: SSIS, T-SQL, and SQL
          Server.
    \end{itemize}
  }

\cventry{2012--2013}
  {Teaching Assistant, Database Systems}
  {Georgia Institute of Technology}
  {Atlanta, GA}
  {}
  {
    \begin{itemize}
      \item Held one-on-one meetings in class of over 240 students to teach
        concepts such as entity-relationship data modeling and gave technical
        help in implementing database-driven applications using MySQL and PHP.
    \end{itemize}
  }

\cventry{2009--2011}
  {Graduate Research Assistant, Materials Simulation}
  {Georgia Institute of Technology}
  {Atlanta, GA}
  {}
  {
    \begin{itemize}
      \item Simulated the mechanical response and texture evolution of
      $\alpha$+$\beta$ titanium alloys via multiscale modeling.
      \item Implemented thermally activated crystal plasticity model,
        microstructure  generation code, and post-processing routines using
        \Fortran, \CC, \Matlab, and Python.
      \item Presented results regularly at the meetings of the Center for
        for Computational Materials Design.
    \end{itemize}
  }

\cventry{2005--2009}
  {Engineering Co-op, Composites Research}
  {Southern Research Institute}
  {Birmingham, AL}
  {}
  {
    \begin{itemize}
      \item Spent five terms performing high-temperature materials research for\
        the aerospace industry.
      \item Investigated the kinetics of phenolic pyrolysis
        via thermogravimetry at temperatures up to \SI{1100}{\celsius}.
      \item Co-authored a report presented at the \nth{56} JANNAF Propulsion
        Meeting.
      \item Designed facility for high-temperature and high-pressure
        thermogravimety and dilatometry.
    \end{itemize}
  }

% \newlength\coursesep
% \setlength{\coursesep}{1.3ex}
% \section{Advanced Coursework}
% \cvitem[\coursesep]{Computer Science}{Adv.\ Operating Systems,
%   Real-Time \& Embedded Systems, Computability and Algorithms, HPC
%   Architecture, Applied Cryptography, Internet Computing, Software
%   Engineering: Analysis \& Testing.}
% \cvitem[\coursesep]{Modeling \& Simulation}{Statistics \& Numerical
%   Methods, Parallel \& Vector Scientific Computing, Adv.\ Constitutive
%   Relations of Solids, Continuum Mechanics, Quantitative
%   Characterization of Materials.}
% \cvitem{Materials Engineering}{Mechanical Behavior of Composites,
%   Thermodynamics of Materials, Kinetics of Phase Transformations,
%   Studies in structure-property relationships of alloys, ceramics,
%   polymers, semiconductors, and composites.}


\section{Scholarships \& Awards}
\cventry{2015}
  {Banner Year Award}
  {for exceptional "stand-out" performance and significant business impact in a calendar year}
  {}
  {}
  {}

\cventry{2014}
  {Extra Mile Award}
  {for extraordinary effort, taking on work outside job scope to help a colleague}
  {}
  {}
  {}

\cventry{2007}
  {Henry Ford Award}
  {for the most outstanding academic record in the junior engineering class}
  {}
  {}
  {}

\cventry{2004}
  {President's Scholarship}
  {Tech's premier merit-based scholarship awarded to approximately 2\% of students}
  {}
  {}
  {}

\end{document}

% Samples...

\section{Experience}
\subsection{Vocational}
\cvitem{description}{item}
\cvitemwithcomment{Language 1}{Skill level}{Comment}
\cvlistitem{Item 1}
\cvlistitem{Item 2}
\cvlistitem{Item 3}
\cvdoubleitem{category 1}{XXX, YYY, ZZZ}{category 4}{XXX, YYY, ZZZ}
\cventry{year--year}{Job title}{Employer}{City}{}{General description
  no longer than 1--2 lines.\newline{}%
  Detailed achievements:%
  \begin{itemize}%
    \item Achievement 1;
    \item Achievement 3.
\end{itemize}}

% change the symbol for lists
\renewcommand{\listitemsymbol}{-~}

\section{Extra 2}
\cvlistdoubleitem{Item 1}{Item 4}
\cvlistdoubleitem{Item 2}{Item 5}%\cite{book1}}
\cvlistdoubleitem{Item 3}{}

\end{document}
