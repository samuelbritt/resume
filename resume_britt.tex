% options: font family ('sans' and 'roman'), standard article options
\documentclass[10pt,letterpaper,roman]{moderncv}

% moderncv themes (load first)
% options: casual, classic, oldstyle, banking
\moderncvstyle{classic}

 % options: blue, orange, green, red, purple, grey, black
% \moderncvcolor{orange}
\definecolor{color0}{rgb}{0,0,0}% black
% \definecolor{color1}{rgb}{0.95,0.55,0.15}% orange
% \definecolor{orange}{rgb}{0.95,0.55,0.15}% orange
% \colorlet{color1}{orange!90!black}
\definecolor{color1}{rgb}{1,.5,0}
\definecolor{color2}{gray}{0.4}% dark grey

\usepackage[osf]{mathpazo}
\usepackage{microtype}
\usepackage{etoolbox}
\usepackage{relsize}
\usepackage{siunitx}
\usepackage[super]{nth}
\usepackage[hscale=0.85,top=.4in,bottom=.3in]{geometry}

% Toggles for various versions. See etoolbox documentation.
\newtoggle{engineering}

% The version
\toggletrue{engineering}

%  default font; use '\sfdefault' for sans serif, '\rmdefault' for
%  roman, or any tex font name
% \renewcommand{\familydefault}{\sfdefault}

% Small caps headings
\renewcommand*\sectionfont{%
  \scshape \lsstyle \fontsize{14}{24}\selectfont%
}

% Italics, not slanted
\renewcommand*{\titlefont}{\large\mdseries\itshape}
\renewcommand*{\addressfont}{\small\mdseries\itshape}

% Pretty ampersand
\let\oldamp\&
\renewcommand\&{{\itshape \oldamp}}

% Pretty Languages
\newcommand\Matlab{\textsc{Matlab}}
\newcommand\Fortran{\textsc{Fortran}}
\newcommand\Unix{\textsc{Unix}}
\newcommand\CC{C%
  \nolinebreak[4]\hspace{-.05em}\raisebox{.29ex}{\relsize{-1}{++}}%
}


% uncomment to suppress page numbering for CVs longer than one page
\nopagenumbers{}

% width of date column
\setlength{\hintscolumnwidth}{1.9cm}

% lists use dashes instead of open bullets
\let\labelitemi\labelitemii

% Short description
\newcommand\cvdesc[2]{%
  \cventry{}%
  {#1}%
  {}%
  {\small #2}%
  {}%
  {}%
}

% Education command
% 1: date
% 2: Degree (M.S., B.S., etc)
% 3: Major
% 4: GPA
% 5: School
% 6: Description
\newcommand\cvedu[6]
  {
    \cventry{#1}
      {#2}
      {#3\hfill\small\upshape{#4}}
      {}
      {}
      {#5
      \ifthenelse{\equal{#6}{}}{}{\\#6}
      }
  }


% Copied from moderncvclassic.cls for the sole purpose of using \itshape
% instead of \slshape in the location slot.
\renewcommand*{\cventry}[7][.25em]{%
  \cvitem[#1]{#2}{%
    {\bfseries#3}%
    \ifthenelse{\equal{#4}{}}{}{, {\itshape#4}}%
    \ifthenelse{\equal{#5}{}}{}{, #5}%
    \ifthenelse{\equal{#6}{}}{}{, #6}%
    .\strut%
    \ifx&#7&%
    \else{\newline{}\begin{minipage}[t]{\linewidth}\small#7\end{minipage}}\fi}%
}

% Github and Linkedin footer items
\input{icons.tex}
\newcommand\smallsymbol[1]{\raisebox{-1pt}{#1\hspace{0pt}}}
\newcommand\githubsymbol{\smallsymbol{\githubpic}}
\newcommand\github[1]{\githubsymbol\httplink{#1}}

\newcommand\linkedinsymbol{\smallsymbol{\linkedinpic}}
\newcommand\linkedin[1]{\linkedinsymbol\httplink{#1}}

% personal data
\firstname{Samuel}
\familyname{Britt}

% All the following are optional
\address{3155 Flowers Rd. S Apt. Q}{Atlanta, GA 30341}
% \email{sam@britts.us}
\email{samuelbritt@gmail.com}
\phone{(205)~515~0618}
% \extrainfo{%
%   \github{github.com/samuelbritt}
%   \\ \footersymbol
%   \linkedin{linkedin.com/in/samuelbritt}
% }
% \mobile{(205)~515~0618}
% \fax{+3~(456)~789~012}
% \homepage{www.johndoe.com}
% % \photo[pic-height][frame-thickness]{pic-file}
% \photo[64pt][0.4pt]{picture}

% Optional
% \quote{Some quote (optional)}
\iftoggle{engineering}{
  % \title{Materials Engineer -- Systems Developer}
  \title{
    \normalsize
    From materials engineering to software development through a
    passion for technology.
  }
  % \title{%\normalsize
  %   Seeking to leverage an engineering background and passion for
  %   technology.
    % to pursue systems software development\\through a passion for
    % technology.
  % }
}{}

\begin{document}
\makecvtitle
\vspace*{-4em}

\section{Education}
\cvedu{2011--\phantom{2013}}
  {M.S.}
  {Computer Science with Specialization in Systems Software}
  {Current GPA: 3.7}
  {Georgia Institute of Technology, Atlanta GA}
  {
    Expected completion: May 3, 2013.
  }

\cvedu{2009--2011}
  {Graduate Research Assistant}
  {Materials Science \& Engineering, Mechanics of Materials}
  {GPA: 3.9}
  {Georgia Institute of Technology, Atlanta GA}
  {
    Modeling and simulation research in the mechanics of
    $\alpha$+$\beta$ titanium alloys.
  }

\cvedu{2004--2009}
  {B.S.\ with Highest Honors}
  {Materials Science \& Engineering}
  {GPA: 4.0}
  {Georgia Institute of Technology, Atlanta GA}
  {}

\section{Skills \& Technologies}
\cvitem{Proficient:}{C, Python (and SciPy), Git, Mercurial, \Unix,
  Linux, Vim, \LaTeX.}
  \cvitem{Familiar:}{\CC, Java, \Matlab, \Fortran, SQL, GDB, OpenMPI,
  Bash, Eclipse, Android development.}

\section{Experience}

\cventry{2012}
  {User-Level Threading Library, Class Project}
  {Georgia Institute of Technology}
  {Atlanta, GA}
  {}
  {
    \begin{itemize}
      \item Developed a C threading library that allowed user-defined
        thread scheduling. The professor selected the library to be
        used in future course offerings.
    \end{itemize}
  }

\cventry{2012}
  {Xen Asynchronous Split-Driver Simulator, Class Project}
  {Georgia Institute of Technology}
  {Atlanta, GA}
  {}
  {
    \begin{itemize}
      \item In a group, developed a multithreaded, multiprocess
        Linux application in C to simulate shared memory ring buffers
        used for asynchronous IO in Xen.  Required use of  semaphores,
        mutexes, and condition variables.
    \end{itemize}
  }

\cventry{2012--2013}
  {Teaching Assistant, Database Systems}
  {Georgia Institute of Technology}
  {Atlanta, GA}
  {}
  {
    \begin{itemize}
      \item Held one-on-one meetings in class of over 240 students to
        teach high-level concepts such as entity-relationship data
        modeling, as well as technical help in implementing
        database-driven applications using MySQL and PHP.
    \end{itemize}
  }

\cventry{2009--2011}
  {Graduate Research Assistant, Materials Simulation}
  {Georgia Institute of Technology}
  {Atlanta, GA}
  {}
  {
    Studied the mechanical response and texture evolution of
    $\alpha$+$\beta$ titanium alloys via multiscale modeling and
    simulation.
    \begin{itemize}
      \item Contributed new, thermally activated constitutive model
        based on unique properties of the titanium microstructure.
      \item Implemented crystal plasticity material model, microstructure
        generation code, and post-processing
        routines using \Fortran, \CC, \Matlab, and Python.
      \item Presented results regularly at the meetings of the Center
        for Computational Materials Design via talks and posters.
      \item Administered the Red Hat cluster used by the research
        group, developing Bash scripts to automate many tasks.
    \end{itemize}
  }

\cventry{2005--2009}
  {Engineering Co-op, Composites Research}
  {Southern Research Institute}
  {Birmingham, AL}
  {}
  {%
    Five terms as an engineering co-op, performing high-temperature
    materials research for the aerospace industry.
    \begin{itemize}%
      \item Investigated the kinetics of phenolic resin pyrolysis
        using isothermal and nonisothermal thermogravimetry at
        temperatures up to \SI{1100}{\celsius}. Co-authored a report
        presented at the \nth{56} JANNAF Propulsion Meeting.
      \item Designed facility for thermogravimety and dilatometry at
        temperatures up to \SI{650}{\celsius} and pressures up to
        \SI{4.15}{\mega\pascal}.
      \item Coordinated effort to develop, build, and test a facility
        capable of tensile permeability tests up to
        \SI{1900}{\celsius}.
      % \item Oversaw operation and maintenance of a permeability
      %   facility testing carbon-phenolic composites up to
      %   \SI{1300}{\celsius}.
    \end{itemize}
  }

\newlength\coursesep
\setlength{\coursesep}{1.3ex}
\section{Advanced Coursework}
\cvitem[\coursesep]{Computer Science}{Adv.\ Operating Systems,
  Real-Time \& Embedded Systems, Computability and Algorithms, HPC
  Architecture, Applied Cryptography, Internet Computing, Software
  Engineering: Analysis \& Testing.}
\cvitem[\coursesep]{Modeling \& Simulation}{Statistics \& Numerical
  Methods, Parallel \& Vector Scientific Computing, Adv.\ Constitutive
  Relations of Solids, Continuum Mechanics, Quantitative
  Characterization of Materials.}
\cvitem{Materials Engineering}{Mechanical Behavior of Composites,
  Thermodynamics of Materials, Kinetics of Phase Transformations,
  Studies in structure-property relationships of alloys, ceramics,
  polymers, semiconductors, and composites.}


\section{Scholarships \& Awards}
\cvdesc{Henry Ford Award}{for the most outstanding academic record in
  the junior engineering class}
% \cvdesc{Wohlford Co-Op Scholarship}{for outstanding contributions to
%   Tech and high scholastic achievement}
% \cvdesc{Chapman-Pentecost Scholarship}{for most outstanding
%   academic record as a junior MSE student}
% \cvdesc{Blount Scholarship}{for outstanding academic performance
%   as an engineering student from Alabama}
\cvdesc{President's Scholarship}{Tech's premier merit-based
  scholarship awarded to approximately \SI{2}{\percent} of students}
\cvdesc{National Merit Scholarship}{awarded to the top
  \SI{0.6}{\percent} of the 1.4 million or so high school
  applicants}
% \cvdesc{S.\ Truett Cathy Scholar Award}{awarded to the top $25$
%   Chick-fil-A employees nationwide for demonstrated excellence in the
%   areas of work, education, community and personal leadership
%   development}
% \cvdesc{Chick-fil-A Leadership Scholarship}{for demonstrated
%   excellence in the areas of work, education, community and
%   personal leadership development}

\end{document}

% Samples...

\section{Experience}
\subsection{Vocational}
\cvitem{description}{item}
\cvitemwithcomment{Language 1}{Skill level}{Comment}
\cvlistitem{Item 1}
\cvlistitem{Item 2}
\cvlistitem{Item 3}
\cvdoubleitem{category 1}{XXX, YYY, ZZZ}{category 4}{XXX, YYY, ZZZ}
\cventry{year--year}{Job title}{Employer}{City}{}{General description
  no longer than 1--2 lines.\newline{}%
  Detailed achievements:%
  \begin{itemize}%
    \item Achievement 1;
    \item Achievement 3.
\end{itemize}}

% change the symbol for lists
\renewcommand{\listitemsymbol}{-~}

\section{Extra 2}
\cvlistdoubleitem{Item 1}{Item 4}
\cvlistdoubleitem{Item 2}{Item 5}%\cite{book1}}
\cvlistdoubleitem{Item 3}{}

\end{document}
